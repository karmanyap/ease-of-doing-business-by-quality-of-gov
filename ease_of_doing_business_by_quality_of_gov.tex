% Options for packages loaded elsewhere
\PassOptionsToPackage{unicode}{hyperref}
\PassOptionsToPackage{hyphens}{url}
%
\documentclass[
]{article}
\usepackage{lmodern}
\usepackage{amssymb,amsmath}
\usepackage{ifxetex,ifluatex}
\ifnum 0\ifxetex 1\fi\ifluatex 1\fi=0 % if pdftex
  \usepackage[T1]{fontenc}
  \usepackage[utf8]{inputenc}
  \usepackage{textcomp} % provide euro and other symbols
\else % if luatex or xetex
  \usepackage{unicode-math}
  \defaultfontfeatures{Scale=MatchLowercase}
  \defaultfontfeatures[\rmfamily]{Ligatures=TeX,Scale=1}
\fi
% Use upquote if available, for straight quotes in verbatim environments
\IfFileExists{upquote.sty}{\usepackage{upquote}}{}
\IfFileExists{microtype.sty}{% use microtype if available
  \usepackage[]{microtype}
  \UseMicrotypeSet[protrusion]{basicmath} % disable protrusion for tt fonts
}{}
\makeatletter
\@ifundefined{KOMAClassName}{% if non-KOMA class
  \IfFileExists{parskip.sty}{%
    \usepackage{parskip}
  }{% else
    \setlength{\parindent}{0pt}
    \setlength{\parskip}{6pt plus 2pt minus 1pt}}
}{% if KOMA class
  \KOMAoptions{parskip=half}}
\makeatother
\usepackage{xcolor}
\IfFileExists{xurl.sty}{\usepackage{xurl}}{} % add URL line breaks if available
\IfFileExists{bookmark.sty}{\usepackage{bookmark}}{\usepackage{hyperref}}
\hypersetup{
  pdftitle={EDA for Ease of Doing Business by Quality of Government},
  pdfauthor={Karmanya Pathak},
  hidelinks,
  pdfcreator={LaTeX via pandoc}}
\urlstyle{same} % disable monospaced font for URLs
\usepackage[margin=1in]{geometry}
\usepackage{graphicx,grffile}
\makeatletter
\def\maxwidth{\ifdim\Gin@nat@width>\linewidth\linewidth\else\Gin@nat@width\fi}
\def\maxheight{\ifdim\Gin@nat@height>\textheight\textheight\else\Gin@nat@height\fi}
\makeatother
% Scale images if necessary, so that they will not overflow the page
% margins by default, and it is still possible to overwrite the defaults
% using explicit options in \includegraphics[width, height, ...]{}
\setkeys{Gin}{width=\maxwidth,height=\maxheight,keepaspectratio}
% Set default figure placement to htbp
\makeatletter
\def\fps@figure{htbp}
\makeatother
\setlength{\emergencystretch}{3em} % prevent overfull lines
\providecommand{\tightlist}{%
  \setlength{\itemsep}{0pt}\setlength{\parskip}{0pt}}
\setcounter{secnumdepth}{-\maxdimen} % remove section numbering
\usepackage{booktabs}
\usepackage{longtable}
\usepackage{array}
\usepackage{multirow}
\usepackage{wrapfig}
\usepackage{float}
\usepackage{colortbl}
\usepackage{pdflscape}
\usepackage{tabu}
\usepackage{threeparttable}
\usepackage{threeparttablex}
\usepackage[normalem]{ulem}
\usepackage{makecell}

\title{EDA for Ease of Doing Business by Quality of Government}
\author{Karmanya Pathak}
\date{2020-09-07}

\begin{document}
\maketitle

\hypertarget{loading-the-packages}{%
\section{Loading the packages}\label{loading-the-packages}}

\hypertarget{research-question}{%
\subsection{Research Question}\label{research-question}}

The QoG dataset explores the quality of government data for the year
2020 from multiple data sources. From multiple categories and data
sources, the focus of this research and report is on the Ease of doing
business.

The research question for this project would be: ``How do business
regulations and enforcements across various economies affect their
respective ease of doing business?''

\hypertarget{describe-data}{%
\subsection{Describe Data}\label{describe-data}}

Out of 1758 variables, this research focuses on 12 business factor
varables and 2 variables identifying the economy or the country.

The following 2 dependent variables represent the country name and their
respective country code to analyze their ease of doing business:

\begin{enumerate}
\def\labelenumi{\arabic{enumi}.}
\item
  cname
\item
  ccode
\end{enumerate}

The following independent variables are considered for analysing ease of
business based on their rating system and the respective scores achieved
by each country for the duration of the study:

\begin{enumerate}
\def\labelenumi{\arabic{enumi}.}
\item
  eob\_dcp16: Dealing with construction permits \protect\hyperlink{1}{1}
\item
  eob\_ec16: Enforcing contracts \protect\hyperlink{2}{2}
\item
  eob\_eob16: Ease of doing business score global
  \protect\hyperlink{3}{3}
\item
  eob\_gc15: Getting credit \protect\hyperlink{4}{4}
\item
  eob\_ge16: Getting electricity \protect\hyperlink{5}{5}
\item
  eob\_ldri: Land dispute resolution index \protect\hyperlink{6}{6}
\item
  eob\_pmi15: Protecting minority investors \protect\hyperlink{7}{7}
\item
  eob\_pt17: Paying taxes \protect\hyperlink{8}{8}
\item
  eob\_ri15: Resolving insolvency \protect\hyperlink{9}{9}
\item
  eob\_rp16: Registering property \protect\hyperlink{10}{10}
\item
  eob\_sab: Starting a business \protect\hyperlink{11}{11}
\item
  eob\_tab16: Trading across borders \protect\hyperlink{12}{12}
\end{enumerate}

The score for each of the above independent variables ranges from 0 to
100, where 0 represents the worst regulatory performance and 100 the
best regulatory performance.

The dataset has 194 rows and 14 columns.

\textbf{Numeric Variables}: 13

ccode, eob\_dcp16, eob\_ec16, eob\_eob16, eob\_gc15, eob\_ge16,
eob\_ldri, eob\_pmi15, eob\_pt17, eob\_ri15, eob\_rp16, eob\_sab,
eob\_tab16

\textbf{Categorical Variables}: 0

\textbf{Factor variables}: 1

cname

\textbf{Total of Missing values}: 101 which is 3.718704 percent of data.

After dropping rows with missing values, the new total number of rows
are 181

Since only 3.718704\% of the observations have missing values, there is
significant data to conduct further research by loading the data frame
without missing values into a new data frame.

The dataset has 181 countries.

\hypertarget{exploratory-data-analysis}{%
\subsection{Exploratory Data Analysis}\label{exploratory-data-analysis}}

Within the scope of this report, only few variables are chosen for
exploration.

The following graphs show the score frequency for starting a business
and trading across borders.

\includegraphics{ease_of_doing_business_by_quality_of_gov_files/figure-latex/Data Exploration-1.pdf}
\includegraphics{ease_of_doing_business_by_quality_of_gov_files/figure-latex/Data Exploration-2.pdf}
\includegraphics{ease_of_doing_business_by_quality_of_gov_files/figure-latex/Data Exploration-3.pdf}

The following graph shows the top 10 countries by ease of doing
business.

\includegraphics{ease_of_doing_business_by_quality_of_gov_files/figure-latex/ease_of_business-1.pdf}

The following graph shows the top 10 countries that have good policies
for taxes.

\includegraphics{ease_of_doing_business_by_quality_of_gov_files/figure-latex/taxes-1.pdf}

The following plot shows the correlation between the variables

\includegraphics{ease_of_doing_business_by_quality_of_gov_files/figure-latex/correlation-1.pdf}

The following graph shows the distribution of data for each variable.

\includegraphics{ease_of_doing_business_by_quality_of_gov_files/figure-latex/boxplot-1.pdf}

The data exploration reveals that there are no missing values and the
variable eob\_ldri Land dispute resolution index has a different data
range.

This column will not be used in future analysis in this research.

Also, variables eob\_sab, eob\_pt17, eob\_tab16 and eob\_dcp16 have
significant frequencies of outliers and decreasing in the said order.

\hypertarget{clustering}{%
\subsection{Clustering}\label{clustering}}

The aim of cluster analysis is to group countries with similar scores of
ease of business and to see which countries fall in each group.

Since the data is already standardized on a scale, there was no need to
scale the data using scale().

Using elbow, silhouette and gap statistics methods to find the optimal
number of clusters, it can be concluded that it is optimal to have 3
clusters for this data frame.

\includegraphics{ease_of_doing_business_by_quality_of_gov_files/figure-latex/final_cluster_plot-1.pdf}
\includegraphics{ease_of_doing_business_by_quality_of_gov_files/figure-latex/final_cluster_plot-2.pdf}
\includegraphics{ease_of_doing_business_by_quality_of_gov_files/figure-latex/final_cluster_plot-3.pdf}
\includegraphics{ease_of_doing_business_by_quality_of_gov_files/figure-latex/final_cluster_plot-4.pdf}

\hypertarget{pca}{%
\subsection{PCA}\label{pca}}

Looking at the data, it is observed that there are more variable than
can be used to form a good analysis. To reduce the number of variables,
Prinicipal Component Analysis is utilized.

For PCA, first the correlation between variables are plotted.

\includegraphics{ease_of_doing_business_by_quality_of_gov_files/figure-latex/cor-1.pdf}

eob\_eob16 has a strong positive linear relationship (+.070) with 7 out
of 10 other numeric variables, namely:

eob\_eob16 and eob\_ec16: 0.71 eob\_eob16 and eob\_ge16: 0.78 eob\_eob16
and eov\_pmi15: 0.7284136 eob\_eob16 and eob\_pt17: 0.71 eob\_eob16 and
eob\_ri15: 0.7769863 eob\_eob16 and eob\_rp16: 0.7848144 eob\_eob16 and
eob\_tab16: 0.7983523

The above graph clearly shows the correlation between some variables are
high whereas others are moderate to weak.

The next steps is to reduce the variable by fitting the data using PCA.

\begin{verbatim}
## Importance of components:
##                           PC1     PC2      PC3      PC4      PC5      PC6
## Standard deviation     44.908 19.0583 15.88461 13.83389 13.25680 11.05088
## Proportion of Variance  0.582  0.1048  0.07281  0.05523  0.05071  0.03524
## Cumulative Proportion   0.582  0.6868  0.75961  0.81484  0.86555  0.90080
##                            PC7     PC8     PC9   PC10    PC11    PC12
## Standard deviation     10.1788 9.56585 8.82618 8.0933 2.02390 1.07883
## Proportion of Variance  0.0299 0.02641 0.02248 0.0189 0.00118 0.00034
## Cumulative Proportion   0.9307 0.95710 0.97958 0.9985 0.99966 1.00000
\end{verbatim}

\begin{verbatim}
##                            PC1       PC2        PC3        PC4
## Afghanistan         -71.710135  18.71333  4.3152229   2.255178
## Albania              23.400133  14.89497 18.8338792   4.109742
## Algeria             -48.663274 -11.99608 26.8998168  -8.907164
## Angola              -81.784466 -16.38438 -7.7797557 -25.668689
## Antigua and Barbuda  -8.585865 -23.07723  0.3957049   5.926952
## Azerbaijan           17.614810 -15.30373 -5.4683719 -23.049409
\end{verbatim}

From the above importance of components, it is deduced that the 1st 4
have a cumulative proportion of 81.48\% of the variance.

This can be further explained using the graphs below.

\includegraphics{ease_of_doing_business_by_quality_of_gov_files/figure-latex/scree_plot-1.pdf}

The Biplot clearly shows that the variable: eob\_ri15, eob\_tab16,
eob\_ge16 and eob\_eob16 have the highest contribution to both PC1 and
PC2.

\includegraphics{ease_of_doing_business_by_quality_of_gov_files/figure-latex/unnamed-chunk-1-1.pdf}

\hypertarget{conclusion}{%
\subsection{Conclusion}\label{conclusion}}

From the above analysis, it can be concluded that business regulations
and enforcements across various economies affect their respective ease
of doing business and they can be grouped into 3 different clusters of
countries from the methods based on the scores of the variables
considered.

The PCA shows us that the whole analysis can be reduced to just 4
components or variables that would reproduce the information with
similar information making the computation faster and easier analysis.

\hypertarget{reference}{%
\subsection{Reference}\label{reference}}

\begin{enumerate}
\def\labelenumi{\arabic{enumi})}
\tightlist
\item
  \url{https://qog.pol.gu.se/data/datadownloads/qogstandarddata}
\item
  Teorell, Jan, Stefan Dahlberg, Sören Holmberg, Bo Rothstein, Natalia
  Alvarado Pachon \& Sofia Axelsson. 2020. The Quality of Government
  Standard Dataset, version Jan20. University of Gothenburg: The Quality
  of Government Institute, \url{http://www.qog.pol.gu.se}
  \url{doi:10.18157/qogstdjan20}
\item
  The QoG Standard Dataset 2020 Codebook
\end{enumerate}

\begin{enumerate}
\def\labelenumi{\roman{enumi})}
\tightlist
\item
  Teorell, Jan, Stefan Dahlberg, Sören Holmberg, Bo Rothstein, Natalia
  Alvarado Pachon \& Sofia Axelsson. 2020. The Quality of Government
  Standard Dataset, version Jan20. University of Gothenburg: The Quality
  of Government Institute, \url{http://www.qog.pol.gu.se}
  \url{doi:10.18157/qogstdjan20}
  \url{http://www.qogdata.pol.gu.se/data/qog_std_jan20.pdf}
\item
  \url{http://www.doingbusiness.org/en/doingbusiness} (The World Bank
  Group, 2019)
\end{enumerate}

\hypertarget{appendix}{%
\subsection{Appendix}\label{appendix}}

1 \{\#1\} eob\_dcp16: Score-Dealing with Construction Permits (DB16-19
methodology) measures the gap between an economy's performance and the
regulatory best practice on the Dealing with Construction permits
indicator components. It is calculated as the simple average of the
scores for Procedures (number), Time (days), Cost (a percentage of the
warehouse value), and the Building Quality Control Index. The score
ranges from 0 to 100, where 0 represents the worst regulatory
performance and 100 the best regulatory performance.

2 \{\#2\} eob\_ec16: Score-Enforcing contracts (DB16 methodology)
measures the gap between an economy's performance and the regulatory
best practice on the Enforcing Contracts indicator components. It is
calculated as the simple average of the scores for Time (days), Cost (\%
of claim value) and Quality of judicial processes index. The score
ranges from 0 to 100, where 0 represents the worst regulatory
performance and 100 the best regulatory performance.

3 \{\#3\} eob\_eob16: Ease of doing business score (DB16 methodology)
captures the gap between an economy's performance and a measure of best
practice across the entire sample of 41 indicators for 10 Doing Business
topics. The score ranges from 0 to 100, where 0 represents the worst
regulatory performance and 100 the best regulatory performance.
Calculating the ease of doing business score for each economy involves
two main steps. In the first step individual component indicators are
normalized to a common unit where each of the 41 component indicators y
(except for the total tax and contribution rate) is rescaled using the
linear transformation (worst - y)/(worst - best). In this formulation
the highest score represents the best regulatory performance on the
indicator across all economies since 2005 or the third year in which
data for the indicator were collected. Both the best regulatory
performance and the worst regulatory performance are established every
five years based on the Doing Business data for the year in which they
are established and remain at that level for the five years regardless
of any changes in data in interim years. In the second step for
calculating the ease of doing business score, the scores obtained for
individual indicators for each economy are aggregated through simple
averaging into one score, first for each topic and then across all 10
topics. For the ease of doing business score (DB16 methodology), the
specific topic scores used are: Score-Starting a business, Score-Dealing
with construction permits (DB16-19 methodology), Score-Getting
electricity (DB16-19 methodology), Score-Registering property (DB16
methodology), Score-Getting credit (DB15-19 methodology),
Score-Protecting minority investors (DB15-19 methodology), Score-Paying
taxes (DB06-16 methodology), Score-Trading across borders (DB16-19
methodology), Score-Enforcing contracts (DB16 methodology),
Score-Resolving insolvency (DB15-19 methodology).

4 \{\#4\} eob\_gc15: Score-Getting credit (DB15-19 methodology) measures
the gap between an economy's performance and the regulatory best
practice on the Getting Credit indicator components. The sub-indicators
are weighted proportionally, according to their contribution to the
total score, with a weight of 60\% assigned to the strength of legal
rights index and 40\% to the depth of credit information index. The
score ranges from 0 to 100, where 0 represents the worst regulatory
performance and 100 the best regulatory performance.

5 \{\#5\} eob\_ge16: Score-Getting electricity (DB16-19 methodology)
measures the gap between an economy's performance and the regulatory
best practice on the Getting Electricity indicator components. It is
calculated as the simple average of the scores for Procedures (number),
Time (days), Cost (\% of income per capita), and Reliability of supply
and transparency of tariff index. The score ranges from 0 to 100, where
0 represents the worst regulatory performance and 100 the best
regulatory performance.

6 \{\#6\} eob\_ldri: Land dispute resolution index (0-8) (DB16-19
methodology) measures the accessibility of conflict resolution
mechanisms and the extent of liability for entities or agents recording
land transactions.

7 \{\#7\} eob\_pmi15: Score-Protecting minority investors (DB15-19
methodology) measures the gap between an economy's performance and the
regulatory best practice on the Protecting Minority Investors indicator
components. It is calculated as the simple average of the scores for
Extent of conflict of interest regulation index (0-10) (DB15-19
methodology) and Extent of shareholder governance index (0-10) (DB15-19
methodology). The score ranges from 0 to 100, where 0 represents the
worst regulatory performance and 100 the best regulatory performance.

8 \{\#8\} eob\_pt17: Score-Paying taxes (DB17-19 methodology) measures
the gap between an economy's performance and the regulatory best
practice on the Paying Taxes indicator components. It is calculated as
the simple average of the scores for Payments (number per year), Time
(hours), Total Tax and Contribution Rate (\% of profits), and Postfiling
index (0-100) (DB17-19 methodology). The score ranges from 0 to 100,
where 0 represents the worst regulatory performance and 100 the best
regulatory performance.

9 \{\#9\} eob\_ri15: Score-Resolving insolvency (DB15-19 methodology)
measures the gap between an economy's performance and the regulatory
best practice on the Resolving Insolvency indicator components. It is
calculated as the simple average of the scores for the Recovery Rate
(cents on the dollar) and the Strength of Insolvency Framework Index
(0-16). The score ranges from 0 to 100, where 0 represents the worst
regulatory performance and 100 the best regulatory performance.

10 \{\#10\} eob\_rp16: Score-Registering Property (DB16 methodology)
measures the gap between an economy's performance and the regulatory
best practice on the Registering Property indicator components. It is
calculated as the simple average of the scores for Procedures (number),
Time (days), Cost (\% of property value), and Quality of land
administration index (0-30) (DB16 methodology). The score ranges from 0
to 100, where 0 represents the worst regulatory performance and 100 the
best regulatory performance.

11 \{\#11\} eob\_sab: Score-Starting a business measures the gap between
an economy's performance and the regulatory best practice on the
Starting a Business indicator components. It is calculated as the simple
average of the scores for Procedures (number), Time (calendar days),
Cost (\% of income per capita), and Paid-in Minimum capital (\% of
income per capita). The scores for the following components are obtained
as such: the score for Procedures (number) is calculated based on the
average of scores for Procedures - Men (number) and Procedures - Women
(number); the score for Time (calendar days) is calculated based on the
average of scores for Time - Men (calendar days) and Time - Women
(calendar days); and the score for Cost (\% of income per capita) is
calculated based on the average of scores for Cost - Men (\% of income
per capita) and Cost - Women (\% of income per capita). The score ranges
from 0 to 100, where 0 represents the worst regulatory performance and
100 the best regulatory performance.

12 \{\#12\} eob\_tab16: Score-Trading across Borders (DB16-19
methodology) measures the gap between an economy's performance and the
regulatory best practice on the Trading across Borders indicator
components. It is calculated as the simple average of the scores for
Time to export: Border compliance (hours), Cost to export: Border
compliance (US dollar), Time to export: Documentary compliance (hours),
Cost to export: Documentary compliance (US dollar), Time to import:
Border compliance (hours), Cost to import: Border compliance (US
dollar), Time to import: Documentary compliance (hours) and Cost to
import: Documentary compliance (US dollar). The score ranges from 0 to
100, where 0 represents the worst regulatory performance and 100 the
best regulatory performance.

\end{document}
